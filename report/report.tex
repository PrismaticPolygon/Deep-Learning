% !TEX TS-program = pdflatex
% !TEX encoding = UTF-8 Unicode

% This is a simple template for a LaTeX document using the "article" class.
% See "book", "report", "letter" for other types of document.

\documentclass[11pt]{article} % use larger type; default would be 10pt

\usepackage[utf8]{inputenc} % set input encoding (not needed with XeLaTeX)

%%% Examples of Article customizations
% These packages are optional, depending whether you want the features they provide.
% See the LaTeX Companion or other references for full information.

%%% PAGE DIMENSIONS
\usepackage{geometry} % to change the page dimensions
\geometry{a4paper} % or letterpaper (US) or a5paper or....
% \geometry{margin=2in} % for example, change the margins to 2 inches all round
% \geometry{landscape} % set up the page for landscape
%   read geometry.pdf for detailed page layout information

\usepackage{graphicx} % support the \includegraphics command and options

% \usepackage[parfill]{parskip} % Activate to begin paragraphs with an empty line rather than an indent

\usepackage{tikz}
\usetikzlibrary{arrows}

%%% PACKAGES
\usepackage{booktabs} % for much better looking tables
\usepackage{array} % for better arrays (eg matrices) in maths
\usepackage{paralist} % very flexible & customisable lists (eg. enumerate/itemize, etc.)
\usepackage{verbatim} % adds environment for commenting out blocks of text & for better verbatim
\usepackage{subfig} % make it possible to include more than one captioned figure/table in a single float
% These packages are all incorporated in the memoir class to one degree or another...

\usepackage{amsmath}
\usepackage{algorithm}
\usepackage{algpseudocode}

%%% HEADERS & FOOTERS
\usepackage{fancyhdr} % This should be set AFTER setting up the page geometry
\pagestyle{fancy} % options: empty , plain , fancy
\renewcommand{\headrulewidth}{0pt} % customise the layout...
\lhead{}\chead{}\rhead{}
\lfoot{}\cfoot{\thepage}\rfoot{}

%%% SECTION TITLE APPEARANCE
\usepackage{sectsty}
\allsectionsfont{\sffamily\mdseries\upshape} % (See the fntguide.pdf for font help)
% (This matches ConTeXt defaults)

%%% ToC (table of contents) APPEARANCE
\usepackage[nottoc,notlof,notlot]{tocbibind} % Put the bibliography in the ToC
\usepackage[titles,subfigure]{tocloft} % Alter the style of the Table of Contents
\renewcommand{\cftsecfont}{\rmfamily\mdseries\upshape}
\renewcommand{\cftsecpagefont}{\rmfamily\mdseries\upshape} % No bold!

%%% END Article customizations

\title{Deep Learning}
\author{ffgt86}
%\date{} % Activate to display a given date or no date (if empty),
         % otherwise the current date is printed 

\begin{document}
\maketitle

\tikzstyle{block} = [draw,minimum size=2em]
\tikzstyle{data} = [draw,shape=circle,minimum size=2em]
% diameter of semicircle used to indicate that two lines are not connected
\def\radius{.7mm}
\tikzstyle{branch}=[fill,shape=circle,minimum size=3pt,inner sep=0pt]
\begin{tikzpicture}[>=latex']
\node[data] at (0,0) (input) {C};
\node[block] at (1.2,0) (l1) {$Linear$};
\draw[->] (input) -- (l1);
\node[block] at (2.8,0) (unpool2) {$Unpool_2$};
\draw[->] (l1) -- (unpool2);
\node[block] at (4.4,0) (dconv3) {$DConv_3$};
\draw[->] (unpool2) -- (dconv3);
\node[block] at (5.9,0) (relu3) {$ReLU$};
\draw[->] (dconv3) -- (relu3);
\node[block] at (7.4,0) (unpool1) {$Unpool_1$};
\draw[->] (relu3) -- (unpool1);
\node[block] at (9,0) (dconv2) {$DConv_2$};
\draw[->] (unpool1) -- (dconv2);
\node[block] at (10.5,0) (relu2) {$ReLU$};
\draw[->] (dconv2) -- (relu2);
\node[block] at (12,0) (dconv1) {$DConv_1$};
\draw[->] (relu2) -- (dconv1);
\node[block] at (13.5,0) (relu1) {$ReLU$};
\draw[->] (dconv1) -- (relu1);
\node[data] at (14.7,0) (output) {$\hat{\mathrm{I}}$};
\draw[->] (relu1) -- (output);
\end{tikzpicture}

Where discrepancies exist between the paper and the source code, I have used the paper. For instance, the TensorFlow version calculcates

\end{document}
